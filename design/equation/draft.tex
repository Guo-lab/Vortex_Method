\documentclass[11pt]{article}
%\input epsf
\usepackage{graphicx}
\usepackage{amsmath}
\usepackage{amssymb}
%\usepackage{multicol}
% -*- Mode: TeX; Modified: "Mon 05 Apr 1999 22:39:15 by dbs"; -*- 

\bibliographystyle{plain}

\oddsidemargin=.25in
\evensidemargin=.25in
\textwidth=6.0in
\topmargin=-0.5in
\textheight=8.5in

%%% `captions.sty' overrides the default figure captions layout.
%%% These variables are used in this style file.
\setlength{\abovecaptionskip}{10pt}
\setlength{\belowcaptionskip}{0pt}

%%% force \subsubsection to be numbered and appear in the table of contents
%%% in report style
\setcounter{secnumdepth}{3}
\setcounter{tocdepth}{3}

%%% In `article' style, remove the chapter number from section numbers and force a
%%% page break at the end of the table of contents.
%%% In `report' style, force \chapter to print ``Part'' instead of ``Chapter'' in headings.
%%dbs odd %% \ifx\chaptername\undefined
%%dbs odd %%   %%%article style has no \chaptername
%%dbs odd %%   \renewcommand{\thesection}{\arabic{section}}%  %%dont put chapter number in the section number
%%dbs odd %%   \renewcommand{\tableofcontents}{%
%%dbs odd %%     \section*{\contentsname
%%dbs odd %%         \@mkboth{%
%%dbs odd %%            \MakeUppercase\contentsname}{\MakeUppercase\contentsname}}%
%%dbs odd %%     \@starttoc{toc}%
%%dbs odd %%     \newpage%
%%dbs odd %%   }
%%dbs odd %% \else
%%dbs odd %%   \renewcommand{\chaptername}{Part}             %%%report style
%%dbs odd %% \fi
%%dbs odd %% 
%%% Macro, variable and command (re)definitions 

\input epsf
%%\input latexsym


%%yikes
%\newcount\ncellsx \newcount\ncellsy   % number of cells in base box
%\newcount\cellsize                    % size of each cell in picture units
%\newcount\halfcell                    % cellsize / 2
%\newcount\circlesize                  % size of base level circles
%\newcount\x \newcount\y               % scratch variables
%
%%%
%%% command: withBaseBox
%%% purpose: draw a picture with a base box and allow other
%%%          Box drawing commands to be executed on the base box
%%% arg1: unit of length for picture (should include length type
%%%       (in,mm,cm,pt,...)), a cell is \cellsize * this value
%%% arg2: number of cells in X (horizontal) direction
%%% arg3: number of cells in Y (vertical) direction
%%% arg4: other drawing commands
%%% NOTE: this opens and closes a picture and defines counters:
%%%       \ncellsx, \ncellsy, \cellsize, \halfcell, \circlesize
%%%
%\newcommand{\withBaseBox}[4]{{
%  % set up constants for this picture
%  \setlength{\unitlength}{#1} \ncellsx=#2 \ncellsy=#3
%  \cellsize=100   %cell size in \unitlength's
%  \halfcell=\cellsize \divide\halfcell by 2
%  \circlesize=\cellsize \multiply\circlesize by 2 \divide\circlesize by 5
%  % make the picture size a little larger than the grid itself
%  \count101=\ncellsx \multiply\count101 by \cellsize \advance\count101 by 10
%  \count102=\ncellsy \multiply\count102 by \cellsize \advance\count101 by 10
%  \begin{picture}(\count101,\count102)
%  % this computes the grid size and draws the grid
%  \count101=\ncellsx \multiply\count101 by \cellsize
%  \count102=\ncellsy \multiply\count102 by \cellsize
%  \put(0,0){\grid(\count101,\count102)(\cellsize,\cellsize)}
%  #4
%  \end{picture}
%}}
%
%%%
%%% command: drawBox
%%% purpose: after a picture has been opened, draw a box with the specified
%%%          lower and upper corners at the specified location with the
%%%          specified refinement ratio
%%% arg1: lower X index (Level 0 index space)
%%% arg2: lower Y index ( " )
%%% arg3: upper X index ( " )
%%% arg4: upper Y index ( " )
%%% arg5: refinement ratio
%%% NOTE: this must be preceded by a \drawBaseBox{}; it leaves the picture open.
%%%
%\newcommand{\drawBox}[5]{{
%  % set the lower corner
%  \x=#1 \multiply\x by \cellsize
%  \y=#2 \multiply\y by \cellsize
%  % set the grid size (upper-lower corners) (in the Level 0 index space)
%  \count151=#3 \advance\count151 by -#1 \advance\count151 by 1 \multiply\count151 by \cellsize
%  \count152=#4 \advance\count152 by -#2 \advance\count152 by 1 \multiply\count152 by \cellsize
%  % refine the grid
%  \count153=\cellsize \divide\count153 by #5
%  % draw the refined grid
%  \put(\x,\y){\grid(\count151,\count152)(\count153,\count153)}
%  % draw a thick outline of the box if the refinement ratio >1
%  \ifnum #5>1
%   { \thicklines\put(\x,\y){\grid(\count151,\count152)(\count151,\count152)}}
%  \fi
%}}
%
%%%
%%% command: drawTag
%%% purpose: draw a Tag in the cell at the specified indices corresponding to
%%%          the specified refinement ratio
%%% arg1: X index of tag
%%% arg2: Y index of tag
%%% arg3: refinement ratio
%%% NOTE: this must be preceded by a \drawBaseBox{}; it leaves the picture open.
%%%
%\newcommand{\drawTag}[3]{{
%  \divide \cellsize by #3 \divide\halfcell by #3
%  \divide\circlesize by #3
%  \x=#1 \multiply\x by \cellsize \advance\x by \halfcell
%  \y=#2 \multiply\y by \cellsize \advance\y by \halfcell
%  \put(\x,\y){\circle*{\circlesize}}  %circle* draws a filled circle
%}}

%
% macros.tex: define macros 
%
\def\aveetoc{Av^{E \rightarrow C}}
\def\avectoe{Av^{C \rightarrow E}}
\def\ten#1#2{$#1 \cdot 10^{#2}$}
\def\prt{\partial}
\def\appx{\sim}
\def\inorm#1{\| #1 \|_\infty}
\def\L2norm#1{\| #1 \|_2}
\def\T{\verb+<+{\bf T}\verb+>+}
\def\Ti#1{\verb+<+{\bf #1}\verb+>+}
\def\pluseq{\mathbin{+\mkern-7mu=}}
\def\minuseq{\mathbin{-\mkern-7mu=}}

\let\realpar=\par%              %%%[NOTE: I don't know what this does. -dbs Apr99]
\newcommand{\Fvec}{\mbox{\boldmath $F$}}
\newcommand{\Gvec}{\mbox{\boldmath $G$}}
\newcommand{\ivec}{{\mbox{\boldmath $i$}}}
\newcommand{\fvec}{{\mbox{\boldmath $f$}}}
\newcommand{\pvec}{\mbox{\boldmath p}}
\newcommand{\vvec}{{\mbox{\boldmath $v$}}}
\newcommand{\uvec}{{\mbox{\boldmath $u$}}}
\newcommand{\evec}{{\mbox{\boldmath $e$}}}
\newcommand{\xvec}{{\mbox{\boldmath $x$}}}
\newcommand{\jvec}{\mbox{\boldmath j}}
\newcommand{\kvec}{\mbox{\boldmath k}}
\newcommand{\normal}{\mbox{\boldmath $n$}}

\newcommand{\vpmd}{{\vbold, \pm, d}}
\newcommand{\vpd}{{\vbold, +, d}}
\newcommand{\vmd}{{\vbold, -, d}}
\newcommand{\zbold}{{\boldsymbol{z}}}
\newcommand{\nbold}{{\boldsymbol{n}}}
\newcommand{\vprime}{{\boldsymbol{v}^{'}}}
\newcommand{\vprimec}{{\boldsymbol{v'_c}}}
\newcommand{\vprimef}{{\boldsymbol{v'_f}}}
\newcommand{\vboldc}{{\boldsymbol{v_c}}}
\newcommand{\vboldf}{{\boldsymbol{v_f}}}
\newcommand{\nref}{{{N}_{ref}}}
\newcommand{\ind}{{\it ind}}

\newcommand{\Abold}{{\mbox{\boldmath $A$}}}
\newcommand{\Bbold}{{\mbox{\boldmath $B$}}}
\newcommand{\Jbold}{{\mbox{\boldmath $J$}}}
\newcommand{\Fbold}{{\mbox{\boldmath $F$}}}
%\newcommand{\ebold}{{\mbox{\boldmath $e$}}}
\newcommand{\ebold}{{\boldsymbol{e}}}
\newcommand{\xbold}{{\boldsymbol{x}}}
\newcommand{\ybold}{{\boldsymbol{y}}}
\newcommand{\betabold}{{\boldsymbol{\beta}}}
\newcommand{\pbold}{{\boldsymbol{p}}}
\newcommand{\qbold}{{\boldsymbol{q}}}
\newcommand{\rbold}{{\boldsymbol{r}}}
\newcommand{\dbold}{{\boldsymbol{d}}}
\newcommand{\jbold}{{\boldsymbol{j}}}
\newcommand{\sbold}{{\boldsymbol{s}}}
\newcommand{\fbold}{{\boldsymbol{f}}}
\newcommand{\wbold}{{\boldsymbol{w}}}
\newcommand{\vbold}{{\boldsymbol{v}}}
\newcommand{\ubold}{{\boldsymbol{u}}}
\newcommand{\ibold}{{\boldsymbol{i}}}
\newcommand{\kbold}{{\boldsymbol{k}}}
\newcommand{\lbold}{{\boldsymbol{l}}}
\newcommand{\Dim}{{\mathbf{D}}}
\newcommand{\Ident}{{\mathbf{I}}}
%\newcommand{\xbold}{{\mbox{\boldmath $x$}}}
%\newcommand{\ibold}{{\mbox{\boldmath $i$}}}
%\newcommand{\ubold}{{\mbox{\boldmath $u$}}}
%\newcommand{\vbold}{{\mbox{\boldmath $v$}}}
%\newcommand{\sbold}{{\mbox{\boldmath $s$}}}
%\newcommand{\jbold}{{\mbox{\boldmath $j$}}}
\newcommand{\beqa}{\begin{eqnarray*}}
\newcommand{\eeqa}{\end{eqnarray*}}
\newcommand{\on}{{\mathbf{~on~}}}

\newcommand{\ebshift}[2]{{#1} \! < \!\! < \! {#2}}
\newcommand{\ebaver}[1]{< \!\! {#1} \!\! >}

\newcommand{\MeshRefine}{\textsf{MeshRefine}{}}
\newcommand{\ChomboVersion}{0.99}
\newcommand{\gmake}{{\tt gmake}}
\newcommand{\Chombo}{{\tt Chombo}}
\newcommand{\AN}{[(U \cdot \nabla)U]^{n+\frac{1}{2}}}
\newcommand{\ChFVersion}{1.6}
\newcommand{\ChF}{{\tt ChF}{}}
\newcommand{\chfpp}{{\tt chfpp}}
\newcommand{\cpp}{{\tt cpp}}
\newcommand{\argv}{$<${\em arg}$>$}
\newcommand{\dimv}{$<${\em dim}$>$}
\newcommand{\lenv}{$<${\em len}$>$}
\newcommand{\compv}{$<${\em comp}$>$}
\newcommand{\ncomp}{$<${\em ncomp}$>$}
\newcommand{\arity}[1]{#1.\fnc{arity}{\relax}}
\newcommand{\Beta}{\beta}
\newcommand{\cdh}{{\cal D_H}}
\newcommand{\cd}{{\cal D}}
\newcommand{\cg}{{\cal G}}
\newcommand{\cq}{{\cal Q}}
\newcommand{\clh}{{\cal L_H}}
\newcommand{\cl}{{\cal L}}
\newcommand{\cph}{{\cal P_H}}
\newcommand{\cp}{{\cal P}}
\newcommand{\del}{\nabla}
\newcommand{\deriv}[2]{\mathop{\frac{\partial #1}{\partial #2}}}
\newcommand{\matDeriv}[2]{\mathop{\frac{D #1}{D #2}}}
\newcommand{\display}[1]{\begin{itemize}\item[]#1\end{itemize}}
\newcommand{\domain}[1]{#1.\fnc{domain}{\relax}}
\newcommand{\Domain}[1]{\ifmmode\mbox{\tt Domain<}#1\mbox{\tt>}\else{\tt Domain<$#1$>}\fi}
\newcommand{\dr}{\Delta r}
\newcommand{\ds}{\displaystyle}
\newcommand{\dth}{\Delta \theta}
\newcommand{\dt}{\Delta t}
\newcommand{\dx}{\Delta x}
\newcommand{\dy}{\Delta y}
\newcommand{\eps}{\epsilon}
\newcommand{\ES}{{\tt EdgeStencil}}
\newcommand{\fab}{{\tt FArrayBox}} 
\newcommand{\fnc}[2]{\ifmmode\mbox{\tt#1(}#2\mbox{\tt)}\else{\tt#1(}$#2${\tt)}\fi}
\newcommand{\fourth}{\frac{1}{4}}
\newcommand{\grad}{\nabla}
\newcommand{\half}{\frac{1}{2}}
\newcommand{\hexm}[2]{{$#1 \cdot 10^{-#2}$}}
\newcommand{\IEFab}[1]{\ifmmode\mbox{\tt IEFab<}#1\mbox{\tt>}\else{\tt IEFab<$#1$>}\fi}
\newcommand{\IG}{{\tt IrregGeom}}
\def \ij {{i  ,j    }}
\newcommand{\ijk  }{{i,j,k}}
\newcommand{\ijkmh}{{i,j,k-\half}}
\newcommand{\ijkmo}{{i,j,k-1}}
\newcommand{\ijkph}{{i,j,k+\half}}
\newcommand{\ijkpo}{{i,j,k+1}}
\newcommand{\ijmh }{{i,j-\half  }}
\newcommand{\ijmhk}{{i,j-\half,k}}
\newcommand{\ijmho}{{r,i,j-\half}}
\newcommand{\ijmht}{{\theta,i,j-\half}}
\newcommand{\ijmok}{{i,j-1,k}}
\newcommand{\ijo  }{{r,i  ,j  }}
\newcommand{\ijp  }{{i  ,j+1  }}
\newcommand{\ijphk}{{i,j+\half,k}}
\newcommand{\ijpho}{{r,i,j+\half}}
\newcommand{\ijpht}{{\theta,i,j+\half}}
\newcommand{\ijpo }{{r,i  ,j+1}}
\newcommand{\ijpok}{{i,j+1,k}}
\newcommand{\ijpt }{{\theta,i  ,j+1}}
\newcommand{\ijt  }{{\theta,i  ,j  }}
\newcommand{\imhj }{{i-\half,j  }}
\newcommand{\imhjk}{{i-\half,j,k}}
\newcommand{\imhjo}{{r,i-\half,j}}
\newcommand{\imhjt}{{\theta,i-\half,j}}
\newcommand{\imojk}{{i-1,j,k}}
\newcommand{\imojmok}{{i-1,j-1,k}}
\newcommand{\imojpok}{{i-1,j+1,k}}
\newcommand{\iphj }{{i+\half,j  }}
\newcommand{\iphjk}{{i+\half,j,k}}
\newcommand{\iphjo}{{r,i+\half,j}}
\newcommand{\iphjt}{{\theta,i+\half,j}}
\newcommand{\ipj  }{{i+1,j    }}
\newcommand{\ipjo }{{r,i+1,j  }}
\newcommand{\ipjp }{{i+1,j+1  }}
\newcommand{\ipjpo}{{r,i+1,j+1}}
\newcommand{\ipjpt}{{\theta,i+1,j+1}}
\newcommand{\ipjt }{{\theta,i+1,j  }}
\newcommand{\ipojk}{{i+1,j,k}}
\newcommand{\ipojmok}{{i+1,j-1,k}}
\newcommand{\ipojpok}{{i+1,j+1,k}}
\newcommand{\lacute}{\mathopen{<}}
\newcommand{\la}{\leftarrow}
%\newcommand{\blbox}{{\tt Box}} 
%\newcommand{\boxarray}{{\tt  BoxArray}} 
\newcommand{\boxlib}{{\tt  BoxLib}} 
\newcommand{\BoxLib}{{\tt  BoxLib}} 
%\newcommand{\BranchNode}{{\tt BranchNode}}
%\newcommand{\cfstencil}{{\tt CFStencil}} 
%\newcommand{\BaseFab}[1]{\ifmmode\mbox{\tt BaseFab<}#1\mbox{\tt>}\else{\tt BaseFab<$#1$>}\fi}
%\newcommand{\FabArray}[1]{\ifmmode\mbox{\tt FabArray<}#1\mbox{\tt>}\else{\tt FabArray<$#1$>}\fi}
%\newcommand{\basefab}[1]{\ifmmode\mbox{\tt BaseFab<}#1\mbox{\tt>}\else{\tt BaseFab<$#1$>}\fi}
%\newcommand{\fabarray}[1]{\ifmmode\mbox{\tt FabArray<}#1\mbox{\tt>}\else{\tt FabArray<$#1$>}\fi}
%\newcommand{\ebamrellip}{{\tt EBAMREllip}} 
%\newcommand{\ebamrlevelmg}{{\tt EBAMRLevelMG}} 
%\newcommand{\ebcellfab}{{\tt EBCellFAB}} 
%\newcommand{\ebmulticellfab}{{\tt EBMultiCellFAB}} 
%\newcommand{\ebfacefab}{{\tt EBFaceFAB}} 
%\newcommand{\ebcfinterp}{{\tt EBCFInterp}} 
%\newcommand{\eblevelcfstencil}{{\tt EBLevelCFStencil}} 
%\newcommand{\eblevelfluxregister}{{\tt EBLevelFluxRegister}} 
%\newcommand{\ebmultifacefab}{{\tt EBMultiFaceFAB}} 
%\newcommand{\ebamrsolver}{{\tt EBAMRSolver}} 
%\newcommand{\eblevelsolver}{{\tt EBLevelSolver}} 
%\newcommand{\eblevelop}{{\tt EBLevelOp}} 
%\newcommand{\eblevelmg}{{\tt EBLevelMG}} 
%\newcommand{\eblib}{{\tt EBLib}} 
%\newcommand{\lowvofs}[2]{\mbox{\rm lowVoFs(#1,#2)}} 
%\newcommand{\highvofs}[2]{\mbox{\rm highVoFs(#1,#2)}} 
%\newcommand{\amrlevel}{{\tt AmrLevel}} 
%\newcommand{\amrpoisson}{{\tt AMRPoisson}} 
%\newcommand{\amrremesh}{{\tt MeshRefine}} 
%\newcommand{\amrsolver}{{\tt AMRSolver}} 
%\newcommand{\infab}{{\tt INFab}} 
%\newcommand{\iefab}{{\tt IEFab}} 
%\newcommand{\amr}{{\tt Amr}} 
%\newcommand{\gridcfstencil}{{\tt GridCFStencil}} 
%\newcommand{\gridhoavecfstencil}{{\tt  GridHOAveCFStencil}} 
%\newcommand{\gridfluxregister}{{\tt GridFluxRegister}} 
%\newcommand{\LeafNode}{{\tt LeafNode}}
%\newcommand{\EBIS}{{\tt EBIndexSpace}}
%\newcommand{\ebis}{{\tt EBIndexSpace}}
%\newcommand{\Edge}{{\tt Edge}}
%\newcommand{\hoavecfstencil}{{\tt  HOAveCFStencil}} 
%\newcommand{\oscfstencil}{{\tt OneSideCFStencil}} 
%\newcommand{\onesidepoiss}{{\tt OneSidePoiss}} 
%\newcommand{\oscfinterp}{{\tt OneSideCFInterp}} 
%\newcommand{\domainghostbc}{{\tt DomainGhostBC}} 
%\newcommand{\boxghostbc}{{\tt BoxGhostBC}} 
%\newcommand{\farraybox}{{\tt FArrayBox}} 
%\newcommand{\onesidecfinterp}{{\tt OneSideCFInterp}} 
%\newcommand{\hoacfinterp}{{\tt HOACFInterp}} 
%\newcommand{\fluxregister}{{\tt FluxRegister}} 
%\newcommand{\INFab}[1]{\ifmmode\mbox{\tt INFab<}#1\mbox{\tt>}\else{\tt INFab<$#1$>}\fi}
%\newcommand{\intvectset}{{\tt IntVectSet}} 
%\newcommand{\intvect}{{\tt IntVect}} 
%%\newcommand{\levelcfstencil}{{\tt LevelCFStencil}} 
%\newcommand{\levelhoavecfstencil}{{\tt  LevelHOAveCFStencil}} 
%\newcommand{\levelfluxregister}{{\tt LevelFluxRegister}} 
%\newcommand{\levelmg}{{\tt LevelMG}} 
%\newcommand{\levelop}{{\tt LevelOp}} 
%\newcommand{\levelopfactory}{{\tt LevelOpFactory}} 
%\newcommand{\levelsolver}{{\tt LevelSolver}} 
%\newcommand{\amrlevelmg}{{\tt AMRLevelMG}} 
%\newcommand{\meshrefine}{{\tt MeshRefine}} 
%\newcommand{\multifab}{{\tt MultiFab}} 
\newcommand{\nmh}{{n - \half}}
\newcommand{\nochapters}{%                        %%% Use this when you're not using \chapter commands.
    \renewcommand \thesection {\arabic{section}}  %%% Redefine section headings to leave out the chapter number.
    \newpage                                      %%% Force a page break because there is
}                                                 %%%  no \chapter command to do it for us.
\newcommand{\Node}{{\tt Node}}
\newcommand{\nph}{{n + \half}}
\newcommand{\NS}{{\tt NodeStencil}}
\newcommand{\pargraph}[1]{\par\noindent{\bf #1.}}
\newcommand{\parmparse}{{\tt ParmParse}} 
\newcommand{\Point}[1]{\ifmmode\mbox{\tt Point<}#1\mbox{\tt>}\else{\tt Point<$#1$>}\fi}
\newcommand{\racute}{\mathclose{>}}
\newcommand{\ra}{\rightarrow}
\newcommand{\rdh}{{\rm D_H}}
\newcommand{\rdo}{{\rm D_O}}
\newcommand{\rd}{{\rm D}}
\newcommand{\RectDomain}[1]{\ifmmode\mbox{\tt RectDomain<}#1\mbox{\tt>}\else{\tt RectDomain<$#1$>}\fi}
\newcommand{\rgo}{{\rm G_O}}
\newcommand{\rg}{{\rm G}}
\newcommand{\rimo}{(R_{out}-R_{in})}
\newcommand{\rlh}{{\rm L_H}}
\newcommand{\rl}{{\rm L}}
\newcommand{\romi}{(R_{out}-R_{in})}
\newcommand{\rph}{{\rm P^H}}
\newcommand{\rpo}{{\rm P^O}}
\newcommand{\rp}{{\rm P}}
\newcommand{\sign}[1]{\ifmmode\mbox{sign}(#1)\else sign(#1)\fi}
\newcommand{\Vector}[1]{\ifmmode\mbox{\tt Vector<}#1\mbox{\tt>}\else{\tt Vector<#1>}\fi}
\newcommand{\vu}{\vec u}

\newcommand{\bi}{\begin{itemize}}
\newcommand{\ei}{\end{itemize}}
\newcommand{\I}{\item}
\newcommand{\D}{\begin{itemize} \item[]}
\newcommand{\bv}{\begin{verbatim}}
\newcommand{\ev}{\end{verbatim}}
\newcommand{\bt}{\bf \tt}

\newcommand{\ijph}{{i,j+\half}}
\newcommand{\ipoj}{{i+1,j}}
\newcommand{\imoj}{{i-1,j}}

\newcommand{\iphjph}{{i+\half,j+\half}}
\newcommand{\iphjmh}{{i+\half,j-\half}}
\newcommand{\imhjph}{{i-\half,j+\half}}
\newcommand{\imhjmh}{{i-\half,j-\half}}
\newcommand{\ipojpo}{{i+1,j+1}}
\newcommand{\ipojmo}{{i+1,j-1}}
\newcommand{\imojpo}{{i-1,j+1}}
\newcommand{\imojmo}{{i-1,j-1}}
\newcommand{\dxot}{\frac {\triangle x}{2}}
\newcommand{\dyot}{\frac {\triangle x}{2}}
\newcommand{\gradxi}{\nabla_\xi}
%\newcommand{\del}{{\nabla \cdot}}
\newcommand{\delxi}{{\nabla_\xi \cdot}}
\newcommand{\AH}{{\del (U \otimes U)^\nph}}
%\renewcommand{\theequation}{\arabic{section}.\arabic{equation}}

%\def\eval#1#2{\left. {#1} \right|_{#2}}
%\def\deriv#1#2{\frac{\partial #1}{\partial #2}}
%\def\derivv#1#2{\frac{\partial^2 #1}{\partial #2^2}}
%\def\derivvv#1#2{\frac{\partial^3 #1}{\partial #2^3}}
%\def\derivvvv#1#2{\frac{\partial^4 #1}{\partial #2^4}}
%\newcommand{\tderiv}[2]{\mathop{\frac{d #1}{d #2}}}
%\def\l({\left(}
%\def\r){\right)}
%\def\prt{\partial}
%\def\appx{\sim}
%\def\inorm#1{\| #1 \|_\infty}
%\def\L2norm#1{\| #1 \|_2}
%%\def\dspace{\vspace{50pt}}
%\def\dspace#1{\noalign{\vskip#1}}
%\def\eqspace{\dspace{8pt}}

%\newcommand{\draft}{
%\special{!userdict begin /bop-hook{gsave 200 30 translate
%65 rotate /Times-Roman findfont 216 scalefont setfont
%0 0 moveto 0.85 setgray (DRAFT) show grestore}def end}
%}


%\def\p{\partial}
%\def\hgapmed{\hspace{25pt}}
%\def\hgapsm{\hspace{15pt}}
%\def\be{ \begin{equation} }
%\def\ee{ \end{equation} }
%\def\bea{ \begin{eqnarray} }
%\def\eea{ \end{eqnarray} }
%\def\beas{ \begin{eqnarray*} }
%\def\eeas{ \end{eqnarray*} }
%\def\dd#1#2{\frac{\partial #1}{\partial #2}}
%\def\({\left(}
%\def\){\right)}
%\def\[{\left[}
%\def\]{\right]}
%\def\u#1#2#3{U_{#1,#2}^{#3}}
%\def\v#1#2#3{V_{#1,#2}^{#3}}
%\def\vtxt#1{V_{{\rm #1}}}
%\def\uh#1#2#3{\hat{U}_{#1,#2}^{#3}}
%\def\vh#1#2{\hat{V}_{#1,#2}}
%\def\utilde{\tilde{U}}
%\def\uT{U^{\rm T}}
%\def\D{\Delta}
%\def\vec#1{{\bf #1}}
%\def\vecx{{\bf x}}
%\def\avg{{\rm avg}}
%\def\calF{{\cal F}}
%\def\calU{{\cal U}}
%\def\calN{{\cal N}}
%\def\calL{{\cal L}}
%\def\cvmgp{{\rm cvmgp}}
%\def\cvmgz{{\rm cvmgz}}
%\def\lamo{\lambda_o}
%\def\lams{\lambda^*}
%\def\dfrac{\displaystyle\frac}
%\def\sign{{\rm sign}}
%\def\O{\Omega}
%\def\ol{\Omega^\ell}
%\def\supers#1{\raise+0.7ex\hbox{\ninerm #1}}

%\def\Um#1#2{{}^{#1}U^{#2}}
%\def\Fm#1#2{{}^{#1}{\cal F}^{#2}}

\newcommand{\vel}{\vec u}

%\newcommand{\Lamdba}{\Lambda}
\newcommand{\alfven}{Alfv\'{e}n }


\oddsidemargin=-.125in
\evensidemargin=-.125in
\textwidth=6.5in
\topmargin=-.5in
\textheight=8.5in
\parskip 3pt
\nonfrenchspacing
\title{Drafts for Project-Particle Methods for Vortex Problems}
\begin{document}
\maketitle

\noindent

We begin with Poisson's equation in continuous form:
\[
    \nabla^2 \psi(\mathbf{x}) = -\rho(\mathbf{x})
\]

The solution using the Green's function is given by:
\[
    \psi(\mathbf{x}) = \int_{\mathbb{R}^2} G(\mathbf{x} - \mathbf{x}') \, \rho(\mathbf{x}') \, d\mathbf{x}'
\]

Solving Poisson's equation numerically on a discrete grid,
means the domain, potential, source, and Green's function are sampled at grid points.
To discretize this:
\vspace{-0.5em}
\begin{itemize}
    \setlength{}{\itemsep=-1pt}
    \item Let the domain be sampled on a uniform 2D grid with spacing \( h \)
    \item Let \( i = (i_x, i_y) \) index the grid points
    \item Let \( \mathbf{x}_i = h \cdot i \) be the physical coordinates
    \item Define \( \rho_j = \rho(\mathbf{x}_j) \) Here, \(G(i - j)\) is the Green's function for the Laplacian in 2D (solution to Poisson's equation) %, and \( G_{i-j} = G(h(i - j)) \)
\end{itemize}

Then, we approximate the integral using a Riemann sum and denote \( \omega^g_j = \rho_j \).
\begin{equation*}
    \begin{aligned}
        \psi_i
        %  & = \psi (\mathbf{x}_i) \approx h^2 \sum_{j \in \mathbb{Z}^2} G(h(i - j)) \, \rho_j \\
        %  & = h^2 \sum_{j \in \mathbb{Z}^2} G(i - j) \, \omega^g_j \text{  (physical units don't matter)} \\
         & = \sum_{j \in \mathbb{Z}^2} G(i - j) \, \omega^g_j \\
        % \psi_i & \approx h^2 \sum_{j \in \mathbb{Z}^2} G(h(i - j)) \, \rho_j                                   \\
    \end{aligned}
\end{equation*}

where \( \omega_j \) represents the value of \( \rho \) at the grid point \( j \), and the sum runs over all grid points \( j \in \mathbb{Z}^2 \). The function \( G(i - j) \) gives the influence of the source at point \( j \) on the potential at point \( i \).

The potential on the grid is given by \( \psi_i \), and the source on the grid is given by \( \omega^g_j \).


Hockney's algorithm utilizes that The Fourier transform of a convolution equals the product of the Fourier transforms.
\begin{equation*}
    \begin{aligned}
        \mathcal{F} \left( f * g \right) & = \mathcal{F}(f) \cdot \mathcal{F}(g)                                 \\
        f * g                            & = \mathcal{F}^{-1} \left[ \mathcal{F}(f) \cdot \mathcal{F}(g) \right]
    \end{aligned}
\end{equation*}



\vspace{.5in}



% You will be implementing parts of a particle-in-cell (PIC) method for vortex dynamics, described below. This is primarily an exercise in more elaborate template programming. Generally speaking, you are integrating an ODE of the form
% \begin{equation}
%     \frac{dX}{dt} = F(t, X)
% \end{equation}
% In this problem set our forcing functions will all be independent of time, so you can ignore the {\tt a\_time} argument, but it is good to have this form available to you when you use RK4 in other projects.
% We will be using the 4th-order explicit Runge-Kutta integration technique to evolved this system of ODEs. In this case $X$ is the class {\tt ParticleSet}.

% The stages of RK4 all require the calculation of quantities of the form
% \begin{equation}
%     k := \Delta t * F(t+\Delta t, X+ k)
% \end{equation}

% Your $F$ operator is an evaluation of everything on the right of the equal sign.  RK4 is built up by various estimates of what the update to $X$ should be, then recombined to cancel out low order error terms to create a stable method with an error in the solution that is $O(\Delta t)^4$.

% Specifically, you will implement the class {\tt ParticlesVelocities}, that has the single member function
% \begin{verbatim}
%   void operator()(ParticleShift& a_result, 
%                   double a_time, 
%                   double a_deltat, 
%                   const ParticleSet& a_X)
% \end{verbatim}

% The input is the current estimate for $k$, {\tt a\_result}, and the output new estimate for $k$ is returned in {\tt a\_result}:
% \begin{gather*}
%     k := \Delta t F(t + \Delta t,X + k)
% \end{gather*}
% Inputs are the time you are to evaluate the function $t+\Delta t$, the timestep to take $\Delta t$, the state at the start of the timestep $X$ in this case {\tt ParticleSet}, and the shift to use to this state in this evaluation of F $k$, represented by the {\tt ParticleShift} class. In the case of our particle method, $F$ has no explicit dependence on the first time argument, but we still have implement our class as if it does, in order to conform to the general RK4 interface.

% \section*{Specific Instructions }
% You are to implement in the directory /src/Particles
% {\tt ParticleVelocities::operator()(ParticleShift\& a\_k, const double\& a\_time, const double\& dt, ParticleSet\& a\_state) }: computes the $k's$ induced on a set of particles by all of the particles in the input {\tt ParticleSet} displaced by the input $k$.
% In addition, you are to implement a driver program that performs the following calculations.
% \begin{enumerate}
%     \item
%           A single particle, with strength $1./h^2$, placed at (i) (.5,.5) , (ii) .4375,5625, (iii) .45,.55 . The number of grid points is given by N = 32, $\Delta t = 1.$; run for 100 time step.
%           In all of these cases, the displacement of the particle should be roundoff, since the velocity induced by a single particle on itself should vanish. In the case of the initial position of (.5,.5) the displacement should be comparable to roundoff. Output: position of the particle after one step
%     \item
%           Two particles: one with strength $1/h^2$ located at (.5,.5), the other with strength 0, located at (.5,.25). The number of grid points is given by N = 32. Run for 300 time steps, $\Delta t = .1$ . The strength 1 particle should not move, while the zero-strength particle should move at constant angular velocity on a circle centered at (.5,.5) of radius .25. Output: graph of the time history of the radius and angle.
%     \item
%           Two particles: located at (.5,.25) and (.5,.75) both with strength $1/h^2$. The number of grid points is given by N = 32. Both particles should move at a  constant angular velocity on a circle centered at (.5,.5) of radius .25. Output: graph of the time history of the radius and angle for both particles.
%     \item
%           Two-patch problem. For each point
%           $ \boldsymbol{i}\in [0 \dots N_p]$, $N_p = 128, 256$, place a particle at the point $\boldsymbol{i} h_p$, $h_p = \frac{1}{N_p}$ provided that
%           \begin{gather*}
%               || \boldsymbol{i}h_p - (.5,.375) || \leq .12 \hbox { or } || \boldsymbol{i} h_p - (.5,.625) || \leq .12 .
%           \end{gather*}
%           The strength of each of the particles should be $h_p^2/h^2$. This corresponds to a pair of patches of vorticity of constant strength. Take the grid spacing $N = 64$. Integrate the solution to time T = 15, plotting the result at least every 1.25 units of time (to make a nifty movie, plot every time step). Set $\Delta t = .025$. We will provide a reference solution against which you can compare yours.
% \end{enumerate}
% By setting {\tt ANIMATION = TRUE} in your makefile, you can produce a pair of plotfiles every time step (particle locations, vorticity field on the grid). The default is to produce a pair of plotfiles at the end of the calculation for the two-patch case.

% \section*{Description of Algorithm for Computing the Velocity Field}
% \begin{enumerate}
%     \item Depositing the charges in the particles on the grid.
%           \begin{gather*}
%               \omega^g_\ibold = \sum \limits_k \omega^k \Psi(\ibold h - \xbold^k)
%           \end{gather*}
%           where the $\xbold^k$'s are the positions of the particles in {\tt a\_state} displaces by the input {\tt a\_k}'s.
%           \begin{gather*}
%               \omega^g \equiv 0
%           \end{gather*}
%           \begin{gather*}
%               \ibold^k = \Big \lfloor \frac{\xbold^k}{h} \Big \rfloor \\
%               \sbold^k =\frac{\xbold^k - \ibold^k h}{h}\\
%               \omega^g_{\ibold^k} += \omega^k (1 - s^k_0)(1-s^k_1) \\
%               \omega^g_{\ibold^k + (1,0)} += \omega^k s^k_0 (1 - s^k_1) \\
%               \omega^g_{\ibold^k + (0,1)} += \omega^k (1 - s^k_0) s^k_1 \\
%               \omega^g_{\ibold^k + (1,1)} += \omega^k s^k_0 s^k_1
%           \end{gather*}
%     \item Convolution with the Green's function to obtain the potential on the grid, using Hockney's algorithm. The Hockney class will be constructed and maintained in {\tt ParticleSet} - all you have to do is call it at the appropriate time.
%           \begin{gather*}
%               \psi_\ibold = \sum \limits_{\jbold \in \mathbb{Z}^2} G(\ibold - \jbold) \omega^g_\jbold
%           \end{gather*}
%     \item Compute the fields on the grid using finite differences.
%           \begin{gather*}
%               \vec{U}^g_\ibold = \Big (\frac{\psi_{\ibold + (0,1)} - \psi_{\ibold - (0,1)}}{2 h} , -\frac{\psi_{\ibold + (1,0)} - \psi_{\ibold - (1,0)}}{2 h} \Big )
%           \end{gather*}
%     \item Interpolate the fields from the grid to the particles.
%           \begin{gather*}
%               \vec{U}^k = \sum \limits_{\ibold \in \mathbb{Z}^2} \vec{U}_\ibold \Psi(\xbold^k - \ibold h)
%           \end{gather*}
%           \begin{gather*}
%               \ibold^k = \Big \lfloor \frac{\xbold^k}{h} \Big \rfloor \\
%               \sbold^k =\frac{\xbold^k - \ibold^k h}{h}
%           \end{gather*}
%           \begin{align*}
%               \vec{U}^k = & \vec{U}^g_\ibold (1 - s^k_0)(1-s^k_1)        \\
%               +           & \vec{U}^g_{\ibold + (1,0)} s^k_0 (1 - s^k_1) \\
%               +           & \vec{U}^g_{\ibold + (0,1)}(1 - s^k_0) s^k_1  \\
%               +           & \vec{U}^g_{\ibold + (1,1)} s^k_0 s^k_1
%           \end{align*}
% \end{enumerate}
% Note that the operator{\tt ParticleVelocities::operator()} requires you to return in {\tt a\_k} the quantities $\Delta t  \vec{U}^k$.

\end{document}

